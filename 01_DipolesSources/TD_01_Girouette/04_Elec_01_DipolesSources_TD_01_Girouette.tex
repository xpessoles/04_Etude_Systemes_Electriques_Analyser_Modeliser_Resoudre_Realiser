\documentclass[10pt]{article}
\input{style/coursHeadings}
\input{style/programHeadings}
\input{style/macros_SII}
\input{style/macros_Titres}
\input{style/macros_Frames}

%Si le boolen xp est vrai : compilation pour xabi
%Sinon compilation Damien
\newboolean{xp}
\setboolean{xp}{true}

\newboolean{prof}
\setboolean{prof}{true}

\newif\ifprof
\proftrue
%\proffalse

\newboolean{td}
\setboolean{td}{true}

\usepackage[%
    pdftitle={},
    pdfauthor={Xavier Pessoles},
    colorlinks=true,
    linkcolor=blue,
    citecolor=magenta]{hyperref}

\def\discipline{Sciences Industrielles de l'Ingénieur}

\def\xxtitre{\ifthenelse{\boolean{xp}}{04 -- Étude des Systèmes Électriques -- Analyser, Modéliser, Résoudre, Réaliser}{}}

\def\xxsoustitre{\ifthenelse{\boolean{xp}}{
Chapitre 1 -- Dipôles, Sources et Circuits électriques}{
}}


\def\xxauteur{\ifthenelse{\boolean{xp}}{
Laurent \textsc{Deschamps}\\
\noindent Xavier \textsc{Pessoles}}{
}}


\def\xxpied{\ifthenelse{\boolean{xp}}{
04 : Systèmes Électriques\\
Ch 1 : Dipôles, Sources et Circuits électriques -- TD 01 -- \ifprof P \else E \fi %
}{
}}

\def\xxcathegorie{\ifthenelse{\boolean{xp}}{
2013 -- 2014 \\
Xavier \textsc{Pessoles}}{
Systèmes électriques - Cours}}





%---------------------------------------------------------------------------


\begin{document}

\ifthenelse{\boolean{xp}}{\input{style/enteteXP}}{\input{style/enteteDI}}



%\renewcommand{\baselinestretch}{1.2}
%\setlength{\parskip}{2ex plus 0.5ex minus 0.2ex}



\begin{comp}
\noindent \textbf{Résoudre :} 
\begin{itemize}
\item 
\end{itemize}

\noindent \textit{Rés -- C1.1 :} 
\end{comp}

\section*{Girouette -- anémomètre de voilier}



\begin{minipage}[c]{.6\linewidth}
On s'intéresse à l'ensemble girouette--anémomètre d'une centrale de navigation monté en tête de mât d'un voilier et plus en particulier à la girouette permettant de connaître l'orientation du vent. 
\begin{obj}
L'objectif est de modéliser le circuit électrique du système afin de pouvoir connaître l'orientation du vent en fonction de la position du potentiomètre rotatif.
\end{obj}

\end{minipage} \hfill
\begin{minipage}[c]{.35\linewidth}
\begin{center}
\includegraphics[width=\textwidth]{images/GirouetteAnemometre}
\end{center}
\end{minipage}

\vspace{.25cm}

\begin{minipage}[c]{.32\linewidth}
\begin{center}
\includegraphics[width=\textwidth]{images/uc}
\end{center}
\end{minipage} \hfill
\begin{minipage}[c]{.32\linewidth}
\begin{center}
\includegraphics[width=\textwidth]{images/contexte}
\end{center}
\end{minipage}  \hfill
\begin{minipage}[c]{.32\linewidth}
\begin{center}
\includegraphics[width=\textwidth]{images/bdd}
\end{center}
\end{minipage} 

\vspace{.25cm}

On donne le diagramme de bloc interne associé au système de mesure de la direction du vent ainsi que le le schéma électrique du potentiomètre rotatif. 

\vspace{.25cm}

\begin{minipage}[c]{.55\linewidth}
\begin{center}
\includegraphics[width=\textwidth]{images/ibd}
\end{center}
\end{minipage}\hfill
\begin{minipage}[c]{.4\linewidth}
\begin{center}
\includegraphics[width=\textwidth]{images/capteur}

\textit{Schématisation du potentiomètre rotatif}
\end{center}
\end{minipage} 

\vspace{.25cm}

On suppose que l'angle du potentiomètre varie de $-\pi$ à $\pi$. On note $R_0= 10 \; k\Omega$ la résistance totale entre $A$ et $C$ et $R'$ la résistance de la piste comprise entre $A$ et $B$. 

\subparagraph{}
\textit{Déterminer l'expression de $R'$ en fonction de $\alpha$ et $R_0$. }
 
\ifprof
\begin{corrige}
Lorsque $\alpha = -\pi$, $R' = 0$; lorsque $\alpha = \pi$, $R' = R_0 $. La variation de résistance est proportionnelle au secteur angulaire, on a donc : 
$$ R' (\alpha)= \dfrac{R_0}{2} + \dfrac{R_0}{2\pi} \alpha  $$
\end{corrige}
\else
\fi

Pour faciliter l'étude de ce capteur, on se ramène au schéma électrique équivalent ci-dessous.

\begin{center}
\includegraphics[width=\textwidth]{images/modele}

\textit{Modélisation du potentiomètre rotatif}
\end{center}

\subparagraph{}
\textit{Quelles doivent être les expressions de $R_1$ et de $R_2$ en fonction de $R$, $R'$ et $R_0$ et les valeurs de $E_1$ et de $E_2$ pour qu'il en soit ainsi ?}
\ifprof
\begin{corrige}

Pour que la modélisation soit conforme au capteur initial, il faut nécessairement que : 
\begin{itemize}
\item $R_1 = R + R_0 - R'$;
\item $R_2 = R + R'$;
\item $E_1 = +6 V$; 
\item $E_2 = -6 V$.
\end{itemize}
\end{corrige}
\else
\fi

On note $E_{Th}$ et $R_{Th}$ les éléments du générateur de Thévenin vus entre le point $B$ et la masse. 

\subparagraph{}
\textit{Exprimer $E_{th}$ en fonction de $E_1$ , $E_2$, $R_1$ et $R_2$, puis en fonction de $R$, $R_0$ et $\alpha$. Exprimer $R_{Th}$ en fonction de $R_1$ et $R_2$ puis en fonction de $R$, $R_0$ et $\alpha$.}
\ifprof
\begin{corrige}
Pour calculer la résistance de Thévenin, on désactive les sources et on calcule la résistance équivalente : 
$$
R_{Th}  = \dfrac{R_1 R_2 }{R_1 + R_2} = \dfrac{\left(R + R_0 - R' \right) \left( R + R'\right)}{R + R_0 - R' + R + R'} 
= \dfrac{R^2 + R_0 R + R_0 R'  - R'^2}{2R + R_0  }
$$

$$
R_{Th}  =  \dfrac{R^2 + R_0 R + R_0 \left(  \dfrac{R_0}{2} + \dfrac{R_0}{2\pi} \alpha \right)  - \left(  \dfrac{R_0}{2} + \dfrac{R_0}{2\pi} \alpha \right) ^2}{2R + R_0  }
 = 
\dfrac{R^2 + R_0 R +   \dfrac{R_0^2}{2} + \dfrac{R_0^2}{2\pi} \alpha   -  \dfrac{R_0^2}{4} - \dfrac{R_0^2}{4\pi^2} \alpha^2 - 2 \dfrac{R_0^2 \alpha }{4\pi}  }{2R + R_0  }
$$


$$
R_{Th}   = 
\dfrac{R^2 + R_0 R +   \dfrac{R_0^2}{4} \left( 1 - \dfrac{\alpha^2}{\pi^2}  \right) }{2R + R_0  }
$$

Pour calculer $E_{Th}$ on utilise le théorème de Millmann et on a : 
$$
E_{Th}
= \dfrac{\dfrac{E_1}{R_1}+\dfrac{E_2}{R_2}}{\dfrac{1}{R_1} + \dfrac{1}{R_2}} 
= \dfrac{E_1 R_2 + E_2 R_1}{R_2 + R_1}
= \dfrac{E_1 \left( R + R'\right) + E_2 \left( R + R_0 - R' \right)}{\left(R + R' \right) +\left( R + R_0 - R'\right)}
$$
$$
E_{Th} \left( \alpha \right)
= \dfrac{E_1 \left( R + \left(\dfrac{R_0}{2} + \dfrac{R_0}{2\pi} \alpha \right)\right) + E_2 \left( R + R_0 - \left(\dfrac{R_0}{2} + \dfrac{R_0}{2\pi} \alpha \right)\right)}{\left(R + \left(\dfrac{R_0}{2} + \dfrac{R_0}{2\pi} \alpha \right)\right) +\left( R + R_0 - \left( \dfrac{R_0}{2} + \dfrac{R_0}{2\pi} \alpha \right)\right)}
= \dfrac{E_1 \left( R + R_0 \dfrac{\pi + \alpha }{2 \pi}  \right) + E_2 \left( R + R_0  \dfrac{\pi - \alpha }{2\pi} \right)}{\left(R + R_0 \dfrac{\pi  + \alpha }{2 \pi} \right) +\left( R + R_0  \dfrac{\pi - \alpha }{2\pi} \right)}
$$

$$
E_{Th} \left( \alpha \right)
= \dfrac{E_1 \left( R + R_0 \dfrac{\pi + \alpha }{2 \pi}  \right) + E_2 \left( R + R_0  \dfrac{\pi - \alpha }{2\pi} \right)}{ 2 R + R_0}
$$
Or, $E_1 = -E_2 = 6\; V$; donc : 

$$
E_{Th} \left( \alpha \right)
= \dfrac{ 6\left( R + R_0 \dfrac{\pi + \alpha }{2 \pi}  \right) -6 \left( R + R_0  \dfrac{\pi - \alpha }{2\pi} \right)}{ 2 R + R_0}
= \dfrac{6\alpha R_0}{\pi \left(2R + R_0 \right)}
$$
\end{corrige}
\else
\fi

\subparagraph{}
\textit{Calculer la valeur des résistances $R$ pour que la tension $V_s$ à vide varie entre $-4\; V$ et $+4\; V$.}
\ifprof
\begin{corrige}
Lorsque le montage est à vide, le courant ne circule pas. L'intensité est donc nulle. En conséquence : 
$$
V_S = E_{Th}
$$

Pour $\alpha=-\pi$, on veut donc $V_S = -4$, en conséquence :
$$
E_{Th}\left( -\pi \right)  = -4 
\Longleftrightarrow
 \dfrac{-6\pi R_0}{\pi \left(2R + R_0 \right)} = -4
\Longleftrightarrow
-6\pi R_0  = -4 \pi \left(2R + R_0 \right)
\Longleftrightarrow
R = \dfrac{R_0}{4}
$$

De même, pour $\alpha=\pi$, on veut donc $V_S = 4$, en conséquence :
$$
E_{Th}\left( \pi \right)  = 4 
\Longleftrightarrow
 \dfrac{6\pi R_0}{\pi \left(2R + R_0 \right)} = 4
\Longleftrightarrow
6\pi R_0  = 4 \pi \left(2R + R_0 \right)
\Longleftrightarrow
R = \dfrac{R_0}{4}
$$


\end{corrige}
\else
\fi

\subparagraph{}
\textit{Tracer les caractéristiques $E_{Th} = f(\alpha)$ et $R_{Th}=g(\alpha)$. Préciser les valeurs minimales et maximales.}
\ifprof
\begin{corrige}
On a donc  $E_{Th} (\alpha) = \dfrac{4\alpha}{ \pi}$ et $R_{Th} (\alpha) =\dfrac{\dfrac{R_0^2}{16} +\dfrac{R_0^2}{4}+   \dfrac{R_0^2}{4} \left( 1 - \dfrac{\alpha^2}{\pi^2}  \right) }{\dfrac{3 R_0}{2}}  
=\dfrac{5R_0 +  4R_0 \left( 1 - \dfrac{\alpha^2}{\pi^2}  \right) }{24 }$
.


\begin{minipage}[c]{.4\linewidth}
\begin{center}
\includegraphics[width=\textwidth]{images/tension}
\end{center}
\end{minipage} \hfill
\begin{minipage}[c]{.56\linewidth}
\begin{center}
\includegraphics[width=\textwidth]{images/resistance}
\end{center}
\end{minipage}

\end{corrige}
\else
\fi





\end{document}

