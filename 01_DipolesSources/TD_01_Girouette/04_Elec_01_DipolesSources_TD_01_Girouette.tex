\documentclass[10pt]{article}
\input{style/coursHeadings}
\input{style/programHeadings}
\input{style/macros_SII}
\input{style/macros_Titres}
\input{style/macros_Frames}

%Si le boolen xp est vrai : compilation pour xabi
%Sinon compilation Damien
\newboolean{xp}
\setboolean{xp}{true}

\newboolean{prof}
\setboolean{prof}{true}

\newif\ifprof
\proftrue
%\proffalse

\newboolean{td}
\setboolean{td}{true}

\usepackage[%
    pdftitle={},
    pdfauthor={Xavier Pessoles},
    colorlinks=true,
    linkcolor=blue,
    citecolor=magenta]{hyperref}

\def\discipline{Sciences Industrielles de l'Ingénieur}

\def\xxtitre{\ifthenelse{\boolean{xp}}{04 -- Étude des Systèmes Électriques -- Analyser, Modéliser, Résoudre, Réaliser}{}}

\def\xxsoustitre{\ifthenelse{\boolean{xp}}{
Chapitre 1 -- Dipôles, Sources et Circuits électriques}{
}}


\def\xxauteur{\ifthenelse{\boolean{xp}}{
Laurent \textsc{Deschamps}\\
\noindent Xavier \textsc{Pessoles}}{
}}


\def\xxpied{\ifthenelse{\boolean{xp}}{
04 : Systèmes Électriques\\
Ch 1 : Dipôles, Sources et Circuits électriques -- TD 01 -- \ifprof P \else E \fi %
}{
}}

\def\xxcathegorie{\ifthenelse{\boolean{xp}}{
2013 -- 2014 \\
Xavier \textsc{Pessoles}}{
Systèmes électriques - Cours}}





%---------------------------------------------------------------------------


\begin{document}

\ifthenelse{\boolean{xp}}{\input{style/enteteXP}}{\input{style/enteteDI}}



%\renewcommand{\baselinestretch}{1.2}
%\setlength{\parskip}{2ex plus 0.5ex minus 0.2ex}



\begin{comp}
\noindent \textbf{Résoudre :} 
\begin{itemize}
\item 
\end{itemize}

\noindent \textit{Rés -- C1.1 :} 
\end{comp}

\section*{Girouette -- anémomètre de voilier}



\begin{minipage}[c]{.6\linewidth}
On s'intéresse à l'ensemble girouette--anémomètre d'une centrale de navigation monté en tête de mât d'un voilier et plus en particulier à la girouette permettant de connaître l'orientation du vent. 
\begin{obj}
L'objectif est de modéliser le circuit électrique du système afin de pouvoir connaître l'orientation du vent en fonction de la position du potentiomètre rotatif.
\end{obj}

\end{minipage} \hfill
\begin{minipage}[c]{.35\linewidth}
\begin{center}
\includegraphics[width=\textwidth]{images/GirouetteAnemometre}
\end{center}
\end{minipage}

\vspace{.25cm}

\begin{minipage}[c]{.32\linewidth}
\begin{center}
\includegraphics[width=\textwidth]{images/uc}
\end{center}
\end{minipage} \hfill
\begin{minipage}[c]{.32\linewidth}
\begin{center}
\includegraphics[width=\textwidth]{images/contexte}
\end{center}
\end{minipage}  \hfill
\begin{minipage}[c]{.32\linewidth}
\begin{center}
\includegraphics[width=\textwidth]{images/bdd}
\end{center}
\end{minipage} 

\vspace{.25cm}

On donne le diagramme de bloc interne associé au système de mesure de la direction du vent ainsi que le le schéma électrique du potentiomètre rotatif. 

\vspace{.25cm}

\begin{minipage}[c]{.55\linewidth}
\begin{center}
\includegraphics[width=\textwidth]{images/ibd}
\end{center}
\end{minipage}\hfill
\begin{minipage}[c]{.3\linewidth}
\begin{center}
%\includegraphics[width=\textwidth]{images/GirouetteAnemometre}
\end{center}
\end{minipage} 

\vspace{.25cm}

On suppose que l'angle du potentiomètre varie de $-\pi$ à $\pi$. On note $R_0= 10 \; k\Omega$ la résistance totale entre $A$ et $C$ et $R'$ la résistance de la piste comprise entre $A$ et $B$. 

\subparagraph{}
\textit{Déterminer l'expression de $R_0$ en fonction de $\alpha$ et $R_0$. }
 
\ifprof
\begin{corrige}
\end{corrige}
\else
\fi

Pour faciliter l'étude de ce capteur, on se ramène au schéma électrique équivalent ci-contre.

\subparagraph{}
\textit{Quelles doivent être les expressions de $R_1$ et de $R_2$ en fonction de $R$, $R'$ et $R_0$ et les valeurs de $E_1$ et de $E_2$ pour qu'il en soit ainsi ?}
\ifprof
\begin{corrige}
\end{corrige}
\else
\fi

On note $E_{Th}$ et $R_{Th}$ les éléments du générateur de Thévenin vus entre le point $B$ et la masse. 

\subparagraph{}
\textit{Exprimer $E_{th}$ en fonction de $E_1$ , $E_2$, $R_1$ et $R_2$, puis en fonction de $R$, $R_0$ et $\alpha$. Exprimer $R_{Th}$ en fonction de $R_1$ et $R_2$ puis en fonction de $R$, $R_0$ et $\alpha$.}
\ifprof
\begin{corrige}
\end{corrige}
\else
\fi

\subparagraph{}
\textit{Calculer la valeur des résistances $R$ pour que la tension $V_s$ à vide varie entre $-4\; V$ et $+4\; V$.}
\ifprof
\begin{corrige}
\end{corrige}
\else
\fi

\subparagraph{}
\textit{Tracer les caractéristiques $E_{Th} = f(\alpha)$ et $R_{Th}=g(\alpha)$. Préciser les valeurs minimales et maximales.}
\ifprof
\begin{corrige}
\end{corrige}
\else
\fi





\end{document}

